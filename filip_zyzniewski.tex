\documentclass[a4paper,sans]{moderncv}
\usepackage[utf8]{inputenc}

\moderncvstyle{casual}
\moderncvcolor{blue}

\usepackage[scale=0.75]{geometry}

% personal information
\name{Filip}{Żyźniewski}
\title{Experienced systems, operations, and software engineer}
\address{address}
\phone[mobile]{phone}

\email{filip.zyzniewski@gmail.com}
%TODO: \homepage{www.zyzniewski.com}
\social[linkedin]{filipzyzniewski}

\begin{document}
\makecvtitle

\section{Key competences}
\cvitem{Technical}{
\begin{itemize}
\item Site Reliability Engineering
\begin{itemize}
\item Cloud deployments (borg)
\item Monitoring (black- and whitebox) and alerting
\item Continuous integration and deployment
\item Reliable oncall engineer for multiple services with a 5 minute reponse SLA
\item Golang programming, code reviews
\item Change management
\item Performance engineering
\item Load balancing
\end{itemize}
\item Operating systems: Linux (Debian)
\item Programming: Go, Python, Bourne shell
\item Version control: Perforce, git
\end{itemize}
Relevant, but rusty (and/or shallow) experience:
C,
SQL,
LDAP,
TFTP,
PXE,
NFS,
iptables,
Subversion,
LXC,
KVM.
}
\cvitem{Social}{
Straightforwardness,
ability to resolve conflicts,
capable of making and keeping contacts,
ability to acquire customer requirements,
lecturing experience.
}
\cvitem{Organizational}{
Management of a company,
ability to plan and fulfill strategies,
working as a member of a team.
}

\section{Experience}
\cventry{2014--now}{Site Reliability Engineer}{Google}{Dublin}{}{
Member of the Access SRE team \newline{}
Responsibilities:
\begin{itemize}
\item Tech Lead for the BeyondCorp troubleshooting portal.
\item Running services implementing BeyondCorp at Google (oncall with 5min SLA, onduty, releases, monitoring)
\item SRE onboarding of new services
\item Project KOL
\end{itemize}
Achievements:
\begin{itemize}
\item BeyondCorp troubleshooting portal 
\begin{itemize}
\item led the project from initial design, through a full launch, to handover for maintenance to a SWE team
\item globally replicated, load balanced RPC and HTTP service that has undergone security and privacy reviews
\item continuously integrated and continuously deployed
\item over 95\% code coverage in tests
\item coordination with multiple stakeholders (SecOps, technical support, ACL system owners, ACL data owners)
\item I have written about it in a ;login: \href{https://www.usenix.org/publications/login/fall2017/escobedo}{article}.
\end{itemize}
\item SRE onboarded two BeyondCorp services: the device inventory and the service mananging all of Google's network level ACLs
\item Mentored (onboarded) multiple new team members
\item Hosted two STEP interns
\item Golang Readability in 3 months.
\item Migrated a fleet of legacy HTTP proxies to a new set of machines spread across 3 continents, without user impact.
\item Shut down an old generation of the BeyondCorp SSH relays (user communications and migration to a new solution, infrastructure decommission)
\end{itemize}
}
\cventry{2012--2013}{Senior Software Performance Engineer}{Sabre Holdings}{Kraków}{}{
Member of the pLab team \newline{}
Responsibilities:
\begin{itemize}
\item
Designing and performing performance tests of
core Sabre infrastructure components.
\item
Analyzing test results, determining performance bottlenecks and their causes.
\item
Supporting existing and building new testing automation tools.
\item
Communicating and coordinating efforts with teams
on three continents on a daily basis.
\item
Establishing technology standards across pLab
as a member of the ``standards committee''.
\end{itemize}
Achievements:
\begin{itemize}
\item
Created a document templating framework for automated generation
of test reports and other related documents.
Technologies:
Python,
YAML,
reStructuredText,
Mako Templates,
SVG,
gnuplot,
dot (graphviz).
\item
Managed to get the team to start caring about internal software quality.
\end{itemize}
}
\cventry{2003-2011}{Co-owner}{Tefnet}{Łódź}{}{
Two person dream team \newline{}
Responsibilities:
\begin{itemize}
\item
Design, implementation, deployment and maintenance of open source based software.
\item
Management of customer relations, providing consulting services and support.
\end{itemize}
Achievements:
\begin{itemize}
\item
\textbf{tefconf} (C library, /usr/bin/tefconf utility, http server,
Python bindings) - system wide configuration tree handling with
intra-referencing and overlay capabilities. The default key database
contained over 6400 keys at the end of 2011.
\item
\textbf{confgen} (Python) – a configuration file generator using
data from tefconf. Used on most of the servers managed by Tefnet.
\item
\textbf{dbx2imap} (Python) – a program that adds missing messages
from Microsoft Outlook / Outlook Express files to appropiate e-mail accounts
via IMAP.
\item
\textbf{pygina} – A Microsoft GINA implementation passing calls to
a Python module.
\item
\textbf{python-tefgina} – a solution employing pygina, tefconf, PyQT and LDAP
to handle domain logons.
\item
\textbf{python-tefrescue} – machine storage backup/restore utilities launched from
a rescue system booted via TFTP and NFS. They contain a pure Python cloop file
format implementation (reading and writing).
\item
\textbf{samba-vfs-python} – samba module dispatching calls to a Python module,
which enabled us to extend samba 3 with Python like it is possible with C.
\item
\textbf{tefvms} – a set of scripts managing kvm and lxc virtual environments
using tefconf.
\item
\textbf{tefrisdrv} – set of Python modules containing (among others) a pure
Python BINL server implementation.
\item
\textbf{vehicle tracking system} - A complete solution for offline vehicle
tracking including gps logging, fuel level monitoring and video recording.
\item
\textbf{thin client system} - A complete thin client system software
(client and server side) supporting audio, encryption, USB key
authentication and session detachement.
\end{itemize}
}
\section{Education}
\cventry{2003--2007}{Master's degree in Computer Science}{University of Łódź}{City}{ISCDED 5A}{
Principal subjects: operating systems, computer networks, databases, logic,
programming languages, algorithms and data structures, TEX and LATEX,
network services, software engineering, state machines and formal languages,
design of computer systems, construction of compilers, artificial intelligence,
middle class servers, genetic algorithms, neural networks, cryptography,
computer graphics, software project management, electronics.
\newline{}
Achievements:
\begin{itemize}
\item
I am the author of the port of Linux 2.6 for the HP Jornada 720 (as a part of
the Jlime project), which is a subject of my master’s degree thesis.
\item Conference lectures: ``Open Source and public education'', ``Graphical terminals``.
\item Conducted the ``Internals of the Unix systems'' classes at the Department of
Mathematics (one term).
\end{itemize}
}

\section{Languages}
\cvitemwithcomment{English}{C1 - Proficient}{I have been working for US based corporations for the last 7 years.}
\cvitemwithcomment{Polish}{C2 - Proficient}{My native tongue.}

\end{document}
