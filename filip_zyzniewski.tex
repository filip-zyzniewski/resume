\documentclass[a4paper,sans]{moderncv}
\usepackage[utf8]{inputenc}

\moderncvstyle{casual}
\moderncvcolor{blue}

\usepackage[scale=0.75]{geometry}

% personal information
\name{Filip}{Żyźniewski}
\title{Experienced systems, operations, and software engineer}
\address{address}
\phone[mobile]{phone}

\email{filip.zyzniewski@gmail.com}
%TODO: \homepage{www.zyzniewski.com}
\social[linkedin]{filipzyzniewski}

\begin{document}
\makecvtitle

\section{Experience}
\cventry{2014--now}{Site Reliability Engineer}{Google}{Dublin}{}{
Member of the Access SRE team \newline{}
TODO
}
\cventry{2012--2013}{Senior Software Performance Engineer}{Sabre Holdings}{Kraków}{}{
Member of the pLab team \newline{}
Responsibilities:
\begin{itemize}
\item
Designing and performing performance tests of
core Sabre infrastructure components.
\item
Analyzing test results, determining performance bottlenecks and their causes.
\item
Supporting existing and building new testing automation tools.
\item
Communicating and coordinating efforts with teams
on three continents on a daily basis.
\item
Establishing technology standards across pLab
as a member of the standards committee.
\end{itemize}
Achievements:
\begin{itemize}
\item
Created a document templating framework for automated generation
of test reports and other related documents.
Technologies and formats employed:
Python,
YAML,
reStructuredText,
Mako Templates,
SVG,
gnuplot,
dot (graphviz).
\item
Managed to get the team to start caring about internal software quality.
\end{itemize}
}
\cventry{2003-2011}{Co-owner}{Tefnet}{Łódź}{}{
Two person dream team \newline{}
Responsibilities:
\begin{itemize}
\item
Design, implementation, deployment and maintenance of open source based software.
\item
Management of customer relations, providing technical support.
\item
Providing consulting services.
\end{itemize}
Achievements:
\begin{itemize}
\item
\textbf{teferp} (written in Python) developing an ERP software package for
one of the biggest authorized service centers in Poland. The system supports
communication with challenging business partners’ information systems.
\item
\textbf{tefconf} (C library, /usr/bin/tefconf utility, http daemon,
Python bindings) - system wide configuration tree handling with
inter-referencing and overlying capabilities. The default key database
contained over 6400 keys at the end of 2011.
\item
\textbf{confgen} (written in Python) – a configuration file generator which
uses data fetched from tefconf to build them. Used on most of the servers
managed by Tefnet.
\item
\textbf{dbx2imap} (written in Python) – a program that adds missing messages
from Microsoft Outlook / Outlook Express files to appropiate e-mail accounts
via IMAP.
\item
\textbf{pygina} – A Microsoft GINA implementation passing calls to
a Python module.
\item
\textbf{python-tefgina} – a solution employing pygina, tefconf, PyQT and LDAP
to handle domain logons.
\item
\textbf{python-tefrescue} – workstation backup/restore utilities launched from
a rescue system booted via TFTP and NFS. They contain a pure Python cloop file
format implementation (reading and writing).
\item
\textbf{samba-vfs-python} – samba module dispatching calls to a Python module,
which enabled us to extend samba 3 with Python like it is possible with C.
\item
\textbf{tefvms} – a set of scripts managing kvm and lxc virtual environments
using tefconf.
\item
\textbf{tefrisdrv} – set of Python modules containing (among others) a pure
Python BINL server implementation.
\item
\textbf{vehicle tracking system} - A complete solution for offline vehicle
tracking including gps logging, fuel level monitoring and video recording.
\item
\textbf{thin client system} - A complete thin client system software
(client and server side) supporting audio, encryption, USB mass storage
authentication and session detachement.
\end{itemize}
}
\section{Education}
\section{Languages}
\cvitemwithcomment{English}{C1 - Proficient}{I have been working for US based corporations for the last 7 years.}
\cvitemwithcomment{Polish}{C2 - Proficient}{My native tongue.}

\section{Interests}
\end{document}
