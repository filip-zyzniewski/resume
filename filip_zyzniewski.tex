\documentclass[a4paper,sans,colorlinks]{moderncv}
\usepackage[utf8]{inputenc}

\moderncvstyle{casual}
\moderncvcolor{blue}

\usepackage[scale=0.75]{geometry}
\AtBeginDocument{\hypersetup{allcolors=gray}}

% personal information
\name{Filip}{Żyźniewski}
\title{Experienced systems, operations and software engineer}
\address{address}
\phone[mobile]{phone}

\email{filip.zyzniewski@gmail.com}
%TODO: \homepage{www.zyzniewski.com}
\social[linkedin]{filipzyzniewski}

\begin{document}

\makecvtitle

\section{Key competences}
\cvitem{Technical}{
\begin{itemize}
\item Site Reliability Engineering
\begin{itemize}
\item Responsible for services handling 0.6MQPS, >100TB of data.
\item Responsible for services processing cardholder data.
\item Oncall with a 5 minute reponse SLO.
\item Cloud deployments
(GCP, Kubernetes, \href{https://kubernetes.io/blog/2015/04/borg-predecessor-to-kubernetes/}{borg}).
\item Monitoring and alerting.
\item Continuous deployment.
\item Change management.
\item Performance engineering.
\item Load balancing.
\item Capacity planning.
\item Data integrity: replication, redundancy, backups.
\end{itemize}
\item Software development
\begin{itemize}
\item Languages: Go, Python, Bourne shell.
\item Version control: git, Perforce.
\item Unit testing, integration testing, continuous integration.
\end{itemize}
\item Operating systems: Linux (Debian).
\end{itemize}
}
\cvitem{Social}{
Working in a team (contributing, leadership, mentoring),
establishing and maintaining good relationships with stakeholders across teams
(communication, negotiation), working in fully remote positions, public speaking.
}
\cvitem{Organizational}{
Planning and execution of strategies,
software product lifecycle management,
running a small business.
}

\newpage

\section{Experience}
\cventry{2019--now}{DevOps Engineer}{\href{https://www.fexco.com/}{Fexco}}{Killorglin, Ireland (remotely)}{}{
Member of the Hospitality solution DevOps team \newline{}
\textbf{Responsibilities}:
\begin{itemize}
\item Development and maintenance of the
\href{https://fexco.com/openconnect/hospitality/}{Hospitality solution}
cloud infrastructure.
\item Running the components of the solution (oncall, onduty, releases, deployments).
\item Mentoring on good Site Reliability Engineering practices.
\end{itemize}
\textbf{Achievements}:
\begin{itemize}
\item Automated publication of the build artifacts.
\item Designed and implemented a Terraform module that can perform a full
environment creation and subsequent deployments without any manual steps.
\item Built the infrastructure of the Merchant Portal - QA, UAT and production environments.
\item Implemented a fully instrumented Firestore exports job (Golang).
\item Designed and implemented an automatic Point in Time Recovery procedure (Firestore, Spanner).
\item Establishing best practices: design documents, observability, instrumentation, SLOs.
\item Successfully working and collaborating in a fully remote position.
\end{itemize}
}
\cventry{2014--2019}{Site Reliability Engineer}{\href{https://www.google.com/}{Google}}{Dublin}{}{
Member of the Access SRE team \newline{}
\textbf{Responsibilities}:
\begin{itemize}
\item Tech Lead for the
\href{https://www.usenix.org/system/files/login/articles/login\_fall17\_08\_escobedo.pdf}{BeyondCorp troubleshooting portal}.
\item Running services implementing
\href{https://cloud.google.com/beyondcorp/}{BeyondCorp} at
Google (oncall, onduty, releases, deployments)
\begin{itemize}
\item root@Google - services that we run can impersonate
any Googler / role account.
\item Critical - if our services break, no work can be done in the company.
\end{itemize}
\item
\href{https://landing.google.com/sre/sre-book/chapters/evolving-sre-engagement-model/}{SRE onboarding}
of new services.
\item measuring
\href{https://landing.google.com/sre/sre-book/chapters/service-level-objectives/}{SLIs},
negotiating SLAs and alerting on SLOs.
\end{itemize}
\textbf{Achievements}:
\begin{itemize}
\item BeyondCorp troubleshooting portal 
\begin{itemize}
\item
Led the project from initial design, through development in a four person team,
to a full launch and handover to a SWE team for maintenance.
\item
A globally replicated, load balanced RPC and HTTP service that has
successfully undergone security, privacy and UX reviews.
\item  Continuously integrated and continuously deployed.
\item Over 30k lines of Golang code.
\item Over 95\% code coverage in tests.
\item
Coordinated with multiple stakeholders (SecOps, technical support,
ACL system owners, ACL data owners, UX engineers).
\item
 I have written about it in a ;login:
\href{https://www.usenix.org/publications/login/fall2017/escobedo}{article}.
\end{itemize}
\item SRE onboarded two services:
\begin{itemize}
\item
the Google
\href{https://storage.googleapis.com/pub-tools-public-publication-data/pdf/44860.pdf}{device inventory}
using Bigtable,
\item the service mananging all of Google's network level ACLs.
\end{itemize}
\item Mentored multiple new team members and hosted two
\href{https://careers.google.com/students/engineering-and-technical-internships/}{STEP} interns.
\item
Golang \href{https://sback.it/publications/icse2018seip.pdf}{Readability}
certified in 3 months with no previous exposure to the language.
\item
 Migrated a fleet of legacy HTTP proxies to a new set of machines
spread across 3 continents without user impact.
\item
Shut down an old generation of the BeyondCorp
\href{https://storage.googleapis.com/pub-tools-public-publication-data/pdf/45728.pdf}{SSH gateways}
(user communications and migration to a new solution,
infrastructure decommission).
\end{itemize}
}
\cventry{2012--2013}{Senior Software Performance Engineer}{\href{https://www.sabre.com}{Sabre Holdings}}{Kraków}{}{
Member of the pLab team \newline{}
\textbf{Responsibilities}:
\begin{itemize}
\item
Designing and performing performance tests of
core Sabre infrastructure components.
\item
Analyzing test results, determining performance bottlenecks and their causes.
\item
Supporting existing and building new testing automation tools.
\item
Communicating and coordinating efforts with teams
on three continents on a daily basis.
\item
Establishing technology standards across pLab
as a member of the ``standards committee''.
\end{itemize}
\textbf{Achievements}: Created a document templating framework for
automated generation of test reports and other related documents.
}
\cventry{2003-2011}{Co-owner}{Tefnet}{Łódź}{}{
Two person dream team \newline{}
\textbf{Responsibilities}:
\begin{itemize}
\item
Design, implementation, deployment and maintenance of
open source based solutions.
\item
Management of customer relations,
providing consulting services and support.
\end{itemize}
\textbf{Achievements}:
\begin{itemize}
\item
A customized Debian system fully parameterized with a
system wide configuration tree with intra-referencing and overlay
capabilities. The default key database contained over 6400 keys at
the end of 2011. We were maintaining our own Debian repository
containing our software.
\item
A netbooted thin client system supporting audio,
encryption, USB key authentication and session detachement.
\item
A suite of Linux services babysitting networks of
Microsoft Windows workstations, supporting the whole
lifecycle starting from an unassisted installation over network.
\item A VM management system supporting KVM and LXC guests.
\item
A complete solution for offline vehicle tracking including gps
logging, fuel level monitoring and video recording.
\end{itemize}
}
\section{Education}
\cventry{2003--2007}{Master's degree in Computer Science}{University of Łódź}{City}{ISCDED 5A}{
\textbf{Achievements}:
\begin{itemize}
\item
I am the author of the
\href{https://git.kernel.org/pub/scm/linux/kernel/git/torvalds/linux.git/tree/arch/arm/mach-sa1100/jornada720.c}{port}
of Linux 2.6 for the HP Jornada 720 (as a part of the Jlime project), which is
a subject of my master’s degree thesis.
\item
Conference lectures:
``\href{http://web.archive.org/web/20080309040727/http://konferencja.sul.uni.lodz.pl:80/index.php?id=3}{Open Source and public education}'',
``\href{http://web.archive.org/web/20051118183822/http://konferencja.sul.uni.lodz.pl:80/program.html}{Graphical terminals}``.
\item
Conducted the ``Internals of the Unix systems'' classes at the
Department of Mathematics (one term).
\end{itemize}
}

\section{Languages}
\cvitemwithcomment{English}{C1 - Proficient}{
I have been working for companies based in the USA and Ireland for the last 8 years.}
\cvitemwithcomment{Polish}{C2 - Proficient}{My native tongue.}

\end{document}
